\documentclass{sigchi-alternate}

\usepackage{tikz}
\usetikzlibrary{calc}
\usepackage{layouts}
\usepackage{lipsum}
\usepackage{printlen}
\uselengthunit{in}
\usepackage[noframe]{showframe}

\title{Sigchi-alternate layout test}
\numberofauthors{3}
\author{
  \alignauthor 1st Author Name\\
    \affaddr{Affiliation}\\
    \affaddr{Address}\\
    \email{e-mail address}\\
    \affaddr{Optional phone number}
  \alignauthor 2nd Author Name\\
    \affaddr{Affiliation}\\
    \affaddr{Address}\\
    \email{e-mail address}\\
    \affaddr{Optional phone number}    
  \alignauthor 3rd Author Name\\
    \affaddr{Affiliation}\\
    \affaddr{Address}\\
    \email{e-mail address}\\
    \affaddr{Optional phone number}}
\toappear{}

\begin{document}
\maketitle

\begin{abstract}
This document is visualizing the layout settings for the sigchi-alternative class.
\end{abstract}

\makeatletter
  \newlength{\myGmlmargin}
  \setlength{\myGmlmargin}{\Gm@lmargin}
  \newlength{\myGmrmargin}
  \setlength{\myGmrmargin}{\Gm@rmargin}
  \newlength{\myGmtmargin}
  \setlength{\myGmtmargin}{\Gm@tmargin}
  \newlength{\myGmbmargin}
  \setlength{\myGmbmargin}{\Gm@bmargin}
  \newlength{\myGmwidth}
  \setlength{\myGmwidth}{\Gm@width}
  \newlength{\myGmheight}
  \setlength{\myGmheight}{\Gm@height}
\makeatother

\definecolor{gray}{rgb}{0.4,0.4,0.4}
\textcolor{gray}{\lipsum[1-7]}

\begin{tikzpicture}[overlay,remember picture]
\tikzstyle{numnode} = [midway,above,fill=white,fill opacity=0.8]
\tikzstyle{pageedge} = [line width=1pt,stealth-stealth]
% Corners
\coordinate (c0) at (current page.north west);
\coordinate (c1) at (current page.north east);
\coordinate (c2) at (current page.south east);
\coordinate (c3) at (current page.south west);
% Corners + margins
\coordinate (m0) at ($ (c0) + (\myGmlmargin, -\myGmtmargin) $);
\coordinate (m1) at ($ (c1) + (-\myGmlmargin, -\myGmtmargin) $);
\coordinate (m2) at ($ (c2) + (-\myGmlmargin, \myGmbmargin) $);
\coordinate (m3) at ($ (c3) + (\myGmlmargin, \myGmbmargin) $);
\draw[line width=2pt,blue] (m0) -- (m1) -- (m2) -- (m3) -- cycle;
% Horizontal margins
\draw[pageedge,red] ($(m0)!0.3!(m3)$) -- ($(m1)!0.3!(m2)$) node [numnode] {\printlength{\myGmwidth}};
\draw[pageedge,red] ($(c0)!0.3205!(c3)$) -- ($(m0)!0.3!(m3)$) node [numnode] {\printlength{\myGmlmargin}};
\draw[pageedge,red] ($(c1)!0.3205!(c2)$) -- ($(m1)!0.3!(m2)$) node [numnode] {\printlength{\myGmrmargin}};
% Vertical margins
\draw[pageedge,red] ($(m0)!0.2!(m1)$) -- ($(m3)!0.2!(m2)$) node [numnode,rotate=90] {\printlength{\myGmheight}};
\draw[pageedge,red] ($(c0)!0.253!(c1)$) -- ($(m0)!0.2!(m1)$) node [numnode,rotate=90] {\printlength{\myGmtmargin}};
\draw[pageedge,red] ($(c3)!0.253!(c2)$) -- ($(m3)!0.2!(m2)$) node [numnode,rotate=90] {\printlength{\myGmbmargin}};
% Column width
\draw[pageedge,magenta] ($(m1)!0.8!(m2)-(\linewidth,0)$) -- ($(m1)!0.8!(m2)$) node [numnode] {\printlength{\linewidth}};
\end{tikzpicture}

\begin{figure*}
 \currentpage
 \twocolumnlayouttrue
 \pagedesign
 \caption{Page layout for this document (via \texttt{layouts} package)}
\end{figure*}

\clearpage
\AddToShipoutPicture*{\ShowFramePicture}


\twocolumn[\centering\Large Testing several length settings]

\section{Text}

\newcommand{\texttest}[2]{
  \item[#1] {#2 The quick brown fox \ldots} \\
  baselineskip = {#2 \the\baselineskip} \\
  1ex = {#2 \the\fontdimen5\font} \\
  abovedisplayskip = {#2 \the\abovedisplayskip} \\
  belowdisplayskip = {#2 \the\belowdisplayskip}
}

\begin{description}
  \texttest{scriptsize}{\scriptsize}
  \texttest{tiny}{\tiny}
  \texttest{small}{\small}
  \texttest{normalsize}{\normalsize}
  \texttest{footnotesize}{\footnotesize}
  \texttest{large}{\large}
  \texttest{Large}{\Large}
  \texttest{LARGE}{\LARGE}
%  \texttest{huge}{\huge}
%  \texttest{Huge}{\Huge}
\end{description}



\section{Lists}

\begin{itemize}
  \item \textbf{Level 1}
  \item itemsep=\the\itemsep
  \item topsep=\the\topsep
  \item parsep=\the\parsep
  \item leftmargin=\the\leftmargin
  \item labelwidth=\the\labelwidth
  \begin{itemize}
    \item \textbf{Level 2}
    \item itemsep=\the\itemsep
    \item topsep=\the\topsep
    \item parsep=\the\parsep
    \item leftmargin=\the\leftmargin
    \item labelwidth=\the\labelwidth
    \begin{itemize}
      \item \textbf{Level 3}
      \item itemsep=\the\itemsep
      \item topsep=\the\topsep
      \item parsep=\the\parsep
      \item leftmargin=\the\leftmargin
      \item labelwidth=\the\labelwidth
      \begin{itemize}
        \item \textbf{Level 4}
        \item itemsep=\the\itemsep
        \item topsep=\the\topsep
        \item parsep=\the\parsep
        \item leftmargin=\the\leftmargin
        \item labelwidth=\the\labelwidth
      \end{itemize}
    \end{itemize}
  \end{itemize}
  \item \ldots
  \item \ldots
  \item \ldots
  \item \ldots
  \item \ldots
  \item \ldots
  \item \ldots
  \item \ldots
  \item \ldots
\end{itemize}

\newpage
\AddToShipoutPicture*{\ShowFramePicture}

\section{Enumerations}

\begin{enumerate}
  \item \textbf{Level 1}
  \item itemsep=\the\itemsep
  \item topsep=\the\topsep
  \item parsep=\the\parsep
  \item leftmargin=\the\leftmargin
  \item labelwidth=\the\labelwidth
  \begin{enumerate}
    \item \textbf{Level 2}
    \item itemsep=\the\itemsep
    \item topsep=\the\topsep
    \item parsep=\the\parsep
    \item leftmargin=\the\leftmargin
    \item labelwidth=\the\labelwidth
    \begin{enumerate}
      \item \textbf{Level 3}
      \item itemsep=\the\itemsep
      \item topsep=\the\topsep
      \item parsep=\the\parsep
      \item leftmargin=\the\leftmargin
      \item labelwidth=\the\labelwidth
      \begin{enumerate}
        \item \textbf{Level 4}
        \item itemsep=\the\itemsep
        \item topsep=\the\topsep
        \item parsep=\the\parsep
        \item leftmargin=\the\leftmargin
        \item labelwidth=\the\labelwidth
      \end{enumerate}
    \end{enumerate}
  \end{enumerate}
  \item \ldots
  \item \ldots
  \item \ldots
  \item \ldots
  \item \ldots
  \item \ldots
  \item \ldots
  \item \ldots
  \item \ldots
\end{enumerate}

\newpage

\section{Other Lengths}

\begin{description}
  \item[floatsep] \the\floatsep
  \item[partopsep] \the\partopsep
  \item[textfloatsep] \the\textfloatsep
  \item[hfuzz] \the\hfuzz
  \item[parskip] \the\parskip
  \item[parindent] \the\parindent
  \item[lineskip] \the\lineskip
  \item[normallineskip] \the\normallineskip
  \item[baselineskip] \the\baselineskip
\end{description}

\end{document}